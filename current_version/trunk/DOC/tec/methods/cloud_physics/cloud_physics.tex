\documentclass[11pt,a4paper]{scrartcl}
\usepackage[latin9]{inputenc}
\usepackage[T1]{fontenc}
\usepackage[english]{babel}
\usepackage[a4paper,top=3.cm,bottom=3.5cm,outer=3cm,inner=3.cm]{geometry}  
\usepackage{booktabs,longtable,tabularx} 
\usepackage{amsmath,amssymb,textcomp}
\usepackage{scrpage2}
\usepackage[bookmarks=true,bookmarksopen=false,bookmarksnumbered=true,colorlinks=false]{hyperref}
\setlength{\parindent}{0pt}

\pagestyle{scrheadings}
\clearscrheadfoot
\cfoot{{\small\sf \thepage}}

%Adapting the references
\newenvironment{bibliographie}[1]
{\begin{thebibliography}{0000}{}
\leftskip=5mm \setlength{\itemindent}{-5mm}#1} 
{\end{thebibliography}}
\makeatletter
\renewcommand\@biblabel[1]{\setlength\labelsep{0pt}}  
\renewcommand\@cite[2]{{#1\if@tempswa , #2\fi}} 
\makeatother


\begin {document}

\begin{center}
{\LARGE\bf\textsf{Introduction to the cloud physics module of PALM}}
\vspace{3.0mm}
\linebreak
{\Large\bf\textsf{\textendash Amendments to the dry version of PALM\textendash}}
\linebreak
\linebreak
 Michael Schr\"{o}ter
\linebreak
13.3.2000
\linebreak
translated and adapted by 
\linebreak
Rieke Heinze 
\linebreak
14.12.2009
\linebreak
\linebreak
last update
\linebreak
\today
\end{center}

\section{Introduction}
The dry version of PALM does not contain any cloud physics. It has been extended 
to account for a nearly complete water cycle and radiation processes:
\vspace{0.2cm}
\newline
{\bf\textsf{Water cycle}}
\begin{itemize}
 \item evaporation/condensation
 \item precipitation
 \item transport of humidity and liquid water 
\end{itemize}
{\bf\textsf{Radiation processes}}
\begin{itemize}
 \item short-wave radiation
 \item long-wave radiation
\end{itemize}
The dynamical processes are covered by advection and diffusion and they are described by the implemented methods. For the consideration of the
thermodynamical processes modifications are necessary in the thermodynamics of PALM. In doing so evaporation and condensation are treated as 
adiabatic processes whereas precipitation and radiation are treated as diabatic processes. In the dry version of PALM the thermodynamic variable 
is the potential temperature $\theta$. The first law of thermodynamics provides the prognostic equation 
for $\theta$. The system of thermodynamic variables has to be extended to deal with phase transitions:
\begin{eqnarray*}
 q_{v} & = &\textnormal{specific humidity to deal with water vapour} \\
 q_{l} & = &\textnormal{liquid water content to deal with the liquid phase}
\end{eqnarray*}
Additionally, dependencies between these variables have to be introduced to describe the changes of state (condensation scheme).
\newline
In introducing the two variables liquid water potential temperature $\theta_{l}$ and total liquid water content $q$ the treatment of the 
thermodynamics is simplified. The liquid water potential temperature $\theta_{l}$ is defined by \cite{betts73} and represents the potential 
temperature attained by evaporating all the liquid water in an air parcel through reversible wet adiabatic descent. In a linearized version 
it is defined as 
\begin{eqnarray}
 \theta_{l} & = & \theta -\frac{L_{v}}{c_{p}}\left(\frac{\theta}{T}\right)q_{l}.
 \label{eq:theta_l}
\end{eqnarray}
For the total water content it is valid:
\begin{eqnarray}
 q & = & q_{v}+q_{l}.
 \label{eq:q}
\end{eqnarray}
The usage of $\theta_{l}$ and $q$ as thermodynamic variables is based on the work of \cite{ogura63} and \cite{orville65}. The advantages of the 
$\theta_l$-$q$ system are discussed by \cite{deardorff76}:
\begin{itemize}
 \item Without precipitation, radiation and freezing processes $\theta_{l}$ and $q$ are conservative quantities (for the whole system).
 \item Therewith, the treatment of grid volumes in which only a fraction is saturated is simplified (sub-grid scale condensation scheme).
 \item Parameterizations of the sub-grid scale fluxes are retained. 
 \item The liquid water content is not a separate variable (storage space is saved).
 \item For dry convection $\theta_{l}$ matches the potential temperature and $q$ matches the specific humidity when condensation is disabled.
 \item Phase transitions do not have to be described as additional terms in the prognostic equations.
\end{itemize}

\section{Model equations}
In combining the prognostic equations for dry convection with the processes for cloud physics the following set of prognostic and diagnostic 
model equations is gained:
\newline
\newline
Equation of continuity
\begin{eqnarray}
 \frac{\partial\overline u_{j}}{\partial x_{j}} & = & 0
 \label{eq:conti}
\end{eqnarray}
Equations of motion
\begin{eqnarray}
 \frac{\partial\overline u_{i}}{\partial t} & = &
  -\frac{\partial \left(\overline u_{j} \overline u_{i}\right)}{\partial x_{j}}  
  -\frac{1}{\rho_{0}}\frac{\partial \overline \pi^{\ast}}{\partial x_{i}} 
  - \varepsilon_{ijk}f_{j}\overline u_{k} - \varepsilon_{i3k}f_{3}u_{\mathrm{g}_{k}}
  + g\frac{\overline\theta_{v}-\langle\overline\theta_{v}\rangle}{\theta_{0}}\delta_{i3} 
  -\frac{\partial\,\tau_{ij}}{\partial x_{j}}
 \label{eq:motion}
\end{eqnarray}
with 
\begin{eqnarray}
 \label{eq:pres}
 \overline \pi^{\ast} & = & \overline p^{\ast} + \frac{2}{3}\rho_{0}\,\overline e \\
 \label{eq:tau}
 \tau_{ij} & = & \overline{u_{j}^{'}u_{i}^{'}} - \frac{2}{3}\overline e\,\delta_{ij} 
\end{eqnarray}
First law of thermodynamics
\begin{eqnarray}
 \frac{\partial\overline \theta_{l}}{\partial t} & = & 
  -\frac{\partial \left(\overline u_{j} \overline \theta_{l}\right)}{\partial x_{j}}
  -\frac{\partial\, \overline{u_{j}^{'}\theta_{l}^{'}}}{\partial x_{j}}  
  +\left(\frac{\partial \overline\theta_{l}}{\partial t}\right)_{\mathrm{RAD}}
  +\left(\frac{\partial \overline\theta_{l}}{\partial t}\right)_{\mathrm{PREC}}
 \label{eq:theta}
\end{eqnarray}
Conservation equation for the total water content
\begin{eqnarray}
 \frac{\partial\overline q}{\partial t} & = & 
  -\frac{\partial \left(\overline u_{j} \overline q\right)}{\partial x_{j}}
  -\frac{\partial\, \overline{u_{j}^{'} q^{'}}}{\partial x_{j}}  
  +\left(\frac{\partial \overline q}{\partial t}\right)_{\mathrm{PREC}}
 \label{eq:total_water}
\end{eqnarray}
Conservation equation for the sub-grid scale turbulent kinetic energy $\overline{e}=\frac{1}{2}\overline{u_{i}^{'2}}$
\begin{eqnarray}
 \frac{\partial \overline e}{\partial t} & = & 
  -\frac{\partial \left(\overline u_{j}\overline e\right)}{\partial x_{j}} 
  -\overline{u_{j}^{'} u_{i}^{'}} \frac{\partial\overline u_{i}}{\partial x_{j}}
  + \frac{g}{\theta_{0}}\overline{u^{'}_{3}\theta_{v}^{'}} 
  -\frac{\partial}{\partial x_{j}}\left\{\overline{u^{'}_{j}\left(e^{'} + \frac{p^{'}}{\rho_{0}}\right)} \right\}- \epsilon
 \label{eq:sgsTKE}
\end{eqnarray}
The virtual potential temperature is needed in equation (\ref{eq:motion}) to calculate the buoyancy term. It is defined by  
e.g. \cite{sommeria77} as 
\begin{eqnarray}
 \overline \theta_{v} &=&
  \left(\overline \theta_{l} + \frac{L_{v}}{c_{p}}\left(\frac{\theta}{T}\right)\overline q_{l}\right)
  \left(1 + 0.61\,\overline q - 1.61\,\overline q_{l}\right).
 \label{eq:theta_v}
\end{eqnarray}
Therewith, the influence of changing in density due to condensation is considered in the buoyancy term. 
\newline
The closure of the model equations is based on the approaches of \cite{deardorff80}:
\begin{eqnarray}
 \label{eq:ujui}
 \overline{u_{j}^{'}u_{i}^{'}} & = & 
  -K_{m}\left(\frac{\partial \overline u_{i}}{\partial x_{j}} 
  + \frac{\partial \overline u_{j}}{\partial x_{i}} \right)
  + \frac{2}{3}\overline e \,\delta_{ij} \\
 \label{eq:ujtheta}
 \overline{u_{j}^{'}\theta_{l}^{'}} & = & 
  -K_{h}\left(\frac{\partial \overline \theta_{l}}{\partial x_{j}}\right) \\
 \label{eq:ujq}
 \overline{u_{j}^{'} q{'}} & = & 
  -K_{h}\left(\frac{\partial \overline q}{\partial x_{j}}\right) \\
 \label{eq:ujp}
 \overline{u^{'}_{j}\left(e^{'} + \frac{p^{'}}{\rho_{0}}\right)} & = &
  -2K_{m}\frac{\partial \overline e}{\partial x_{j}} \\
 \label{eq:u3theta_v}
 \overline{u_{3}^{'}\theta_{v}^{'}} & = &
  K_{1} \,\overline{u_{3}^{'}\theta_{l}^{'}}
  + K_{2} \,\overline{u_{3}^{'} q^{'}} \\
 \label{eq:km} 
  K_{m} & = & 0.1\,l\,\sqrt{\overline e} \\
 \label{eq:kh} 
  K_{h} & = & \left(1+2\frac{l}{\Delta}\right)K_{m} \\
 \label{eq:epsilon} 
  \epsilon & = & \left(0.19 + 0.74\,\frac{l}{\Delta}\right)\,\frac{\overline e^{\frac{3}{2}}}{l}
\end{eqnarray}
with 
\begin{eqnarray}
 l = \begin{cases}
  \min\left(\Delta,\,  0.7\,d,\, 0.76\, \sqrt{\overline e}\,\left(\frac{g}{\theta_{0}}\frac{\partial \overline\theta_{v}}
  {\partial z}\right)^{-\frac{1}{2}}\right) & , \quad \frac{\partial \overline\theta_{v}}{\partial z} > 0\\
  \min\left(\Delta,\, 0.7\, d\right)   & , \quad \frac{\partial \overline\theta_{v}}{\partial z} \leq 0 
  \end{cases}
 \label{eq:l}
\end{eqnarray}
and 
\begin{eqnarray}
 \Delta & = & \left(\Delta x \Delta y \Delta z\right)^{1/3}
 \label{eq:delta}
\end{eqnarray}
At the lower boundary Monin-Obukhov similarity theory is valid ( $\overline{w^{'}q^{'}}=u_{\ast}q_{\ast}$). 
\newline
\cite{cuijpers93} for example define the coefficients $K_{1}$ and $K_{2}$ as follows:\newline
{\bf\textsf{in unsaturated air:}}
\begin{eqnarray}
 \label{eq:K_1}
  K_{1} & = & 1.0 + 0.61\, \overline q \\
 \label{eq:K_2}
  K_{2} & = &  0.61\, \overline{\theta}
\end{eqnarray}
{\bf\textsf{in saturated air:}}
\begin{eqnarray}
 \label{eq:K_1_sat}
  K_{1} & = & \frac{1.0-\overline q + 1.61\,\overline q_{s}\left(1.0 + 0.622\frac{L_{v}}{RT}\right)}
  {1.0 + 0.622\frac{L_{v}}{RT}\,\frac{L_{v}}{c_{p}T}\overline q_{s}}  \\
 \label{eq:K_2_sat}
  K_{2} & = &  \overline{\theta}\left(\left(\frac{L_{v}}{c_{p}T}\right)K_{1}-1.0\right)
\end{eqnarray}
The saturation value of the specific humidity comes from the truncated Taylor expansion of $q_{s}(T)$:
\begin{eqnarray}
 q_{s}(T) = q_{s} = q_{s}\left(T_{l}\right) 
 + \left(\frac{\partial q_{s}}{\partial T}\right)_{T=T_{l}} (T-T_{l}). 
 \label{eq:qs1}
\end{eqnarray}
Using the Clausius-Clapeyron equation 
\begin{eqnarray}
 \left(\frac{\partial q_{s}}{\partial T}\right)_{T=T_{l}} & = & 
  0.622\frac{L_{v}q_{s}(T_{l})}{R\,T_{l}^{2}}
 \label{eq:clausius}
\end{eqnarray}
with 
\begin{eqnarray}
  T = T_{l} + \frac{L_{v}}{c_{p}}q_{l} \qquad \textnormal{respectively}  \qquad q_{l} = q - q_{s}
 \label{eq:T}
\end{eqnarray}
gives
\begin{eqnarray}
  \overline q_{s}(T) = \overline q_{s}(\overline T_{l})\frac{\left(1.0+\beta\,\overline q\right)}
    {1.0 + \beta\, \overline q_{s}(\overline{T_{l}})}.
 \label{eq:qs2}
\end{eqnarray}
Whereas
\begin{eqnarray}
  \overline q_{s}(\overline T_{l}) = 0.622\frac{\overline e_{s}(\overline T_{l})}
    {p_{0}(z)-0.378\,\overline e_{s}(\overline T_{l})}
 \label{eq:qs3}
\end{eqnarray}
and
\begin{eqnarray}
 \beta = 0.622\left(\frac{L_{v}}{R\,\overline T_{l}}\right) \left(\frac{L_{v}}{c_{p}\,\overline T_{l}}\right).
 \label{eq:beta}
\end{eqnarray}
The actual liquid water temperature is defined as 
\begin{eqnarray}
 \overline T_{l} = \left(\frac{p_{0}(z)}{p_{0}(z=0)}\right)^{\kappa} \overline\theta_{l}
 \label{eq:T_l}
\end{eqnarray}
with $p_{0}(z=0) = 1000$\,hPa. The value of the saturation vapour pressure at the temperature $\overline T_{l}$ is 
calculated in the same way as in \cite{bougeault82}:
\begin{eqnarray}
 \overline e_{s}(\overline T_{l}) = 610.78 \exp\left(
  17.269\frac{\overline T_{l}-273.16}{\overline T_{l}-35.86}\right).
 \label{eq:es}
\end{eqnarray}
The hydrostatic pressure $p_{0}(z)$ is given by \cite{cuijpers93}:
\begin{eqnarray}
 p_{0}(z) = p_{0}(z=0)\left(\frac{T_{\mathrm{ref}}(z)}{T_{0}}\right)^{c_{p}/R}
 \label{eq:p_0}
\end{eqnarray}
with
\begin{eqnarray}
 T_{\mathrm{ref}}(z) = T_{0} - \frac{g}{c_{p}} z.
 \label{eq:T_ref}
\end{eqnarray}
The pressure is calculated once at the beginning of a simulation and remains unchanged. For the reference temperature at the earth surface $T_{0}$ 
the initial surface temperature is applied. The ratio of the potential and the actual temperature is given by:
\begin{eqnarray}
 \frac{\theta}{T} = \left(\frac{p_{0}(z=0)}{p_{0}(z)}\right)^{\kappa}.
 \label{eq:ratio}
\end{eqnarray}
The liquid water content $q_{l}$ is needed for the calculation of the virtual potential temperature (eq. (\ref{eq:theta_v})). It is 
calculated from the difference of the total water content at a single grid point and the saturation value at this grid point:
\begin{eqnarray}
 \overline q_{l} = 
 \begin{cases}
   \overline q - \overline q_{s}(\overline T) & 
   \textnormal{if} \quad \overline q > \overline q_{s}(\overline T) \\
   0  & \textnormal{else}
  \end{cases}
 \label{eq:q_l}
\end{eqnarray}
With this approach a grid volume is either completely saturated or completely unsaturated. The values of the cloud cover of a grid volume 
can only become 0 or 1 (\textsl{0\%-or100\% scheme}).

\section{Parameterization of the source terms in the conservation equations}
\subsection{Radiation model}
The source term for radiation processes is parameterized via the scheme of effective emissivity which is based on \cite{cox76}:
\begin{eqnarray}
 \left(\frac{\partial \overline\theta_{l}}{\partial t}\right)_{\mathrm{RAD}} & = &
  -\frac{\theta}{T}\frac{1}{\rho \,c_{p}\,\Delta z}\left[\Delta F(z^{+})-\Delta F(z^{-})\right]
 \label{eq:radiation_term}
\end{eqnarray}
$\Delta F$ describes the difference between upward and downward irradiance at the grid point above ($z^{+}$) and below ($z^{-}$) 
the level in which $\theta_{l}$ is defined.
\newline
The upward and downward irradiance $F\textnormal{\textuparrow}$ and $F\textnormal{\textdownarrow}$ are defined as follows:
\begin{eqnarray}
 \label{eq:F_up}
  F\textnormal{\textuparrow}(z) & = &
  B(0) + \varepsilon\textnormal{\textuparrow}(z,0)\left(B(z)-B(0)\right) \\
 \label{eq:F_down} 
  F\textnormal{\textdownarrow}(z) & = &
  F\textnormal{\textdownarrow}(z_{\mathrm{top}})
  + \varepsilon\textnormal{\textdownarrow}(z,z_{\mathrm{top}})\left(B(z)-F\textnormal{\textdownarrow}(z_{\mathrm{top}})\right)
\end{eqnarray}
$F\textnormal{\textdownarrow}(z_{\mathrm{top}})$ describes the impinging irradiance at the upper boundary of the model domain which has to be 
prescribed. $B(0)$ and $B(z)$ represent the black body emission at the ground and the height $z$ respectively. 
$\varepsilon\textnormal{\textuparrow}(z,0)$ and $\varepsilon\textnormal{\textdownarrow}(z,z_{\mathrm{top}})$ stand for the effective 
cloud emissivity of the liquid water between the ground and the level $z$ and between $z$ and the upper boundary of the model domain 
$z_{\mathrm{top}}$ respectively. They are defined as 
\begin{eqnarray}
 \label{eq:epsilon_up}
  \varepsilon\textnormal{\textuparrow}(z,0) & = & 
   1- \exp\left(-a\cdot LWP(0,z)\right)\\
 \label{eq:epsilon_down} 
  \varepsilon\textnormal{\textdownarrow}(z,z_{\mathrm{top}}) & = & 
   1- \exp\left(-b\cdot LWP(z,z_{\mathrm{top}})\right)
\end{eqnarray}
$LWP(z_{1},z_{2})$ describes the liquid water path which is the vertically added content of liquid water above each grid column: 
\begin{eqnarray}
 LWP(z_{1},z_{2}) & = & \int_{z_{1}}^{z_{2}}\mathrm{dz}\,\rho\cdot\overline q_{l}.
 \label{eq:LWP}
\end{eqnarray}
$a$ and $b$ are called mass absorption coefficients. Their empirical values are based on \linebreak
\cite{stephans78} with $a=130\,\mathrm{m^{2}kg^{-1}}$ and $b=158\,\mathrm{m^{2}kg^{-1}}$.
\newline
The assumptions for the validity of this parameterization are:
\begin{itemize}
 \item Horizontal divergences in radiation are neglected.
 \item Only absorption and emission of long-wave radiation due to water vapour and cloud droplets is considered.
 \item The atmosphere is assumed to have constant in-situ temperature above and below the regarded level except for the earth surface.
\end{itemize}

\subsection{Precipitation model}
The source term for precipitation processes is parameterized via a simplified scheme of \cite{kessler69}:
\begin{eqnarray}
 \left(\frac{\partial \overline q}{\partial t}\right)_{\mathrm{PREC}} & = &
  \begin{cases}
   \left(\overline q_{l}-\overline q_{l_{\mathrm{crit}}}\right)/ \tau & 
   \quad\overline q_{l} > \overline q_{l_{\mathrm{crit}}} \\
   0  & \quad\overline q_{l} \leq \overline q_{l_{\mathrm{crit}}}
  \end{cases}
 \label{precip_term_q}
\end{eqnarray}
The precipitation leaves the grid volume immediately if the threshold of the liquid water content 
$\overline q_{l_{\mathrm{crit}}}=0.5\,\mathrm{g/kg}$ is exceeded. Hence, evaporation of the rain drops does not occur.
$\tau$ is a time scale with a value of 1000\,s.
\newline
The influence of the precipitation on the temperature is as follows:
\begin{eqnarray}
 \left(\frac{\partial \overline\theta_{l}}{\partial t}\right)_{\mathrm{PREC}} & = &
  -\frac{L_{v}}{c_{p}}\left(\frac{\theta}{T}\right)\left(\frac{\partial \overline q}{\partial t}\right)_{\mathrm{PREC}}
 \label{precip_term_pt}
\end{eqnarray}

\section*{List of symbols}
\setlength{\extrarowheight}{0.8mm}
\begin{longtable}{p{2.5cm} p{9.0cm} p{2.5cm}} 
\toprule
\addlinespace
\textbf{Variable} & \textbf{Description} &\textbf{Value} \\
\midrule
 $B$ & black body radiation & \\
 $c_{p}$ & heat capacity for dry air with p=const  & $1005\,\mathrm{J\,K^{-1}kg^{-1}}$  \\
 $d$ & normal distance to the nearest solid surface & \\
 $\overline e$ & sub-grid scale turbulent kinetic energy  & \\
 $e_{s}$ & saturation vapour pressure  & \\
 $f_{i}$ & Coriolis parameter $i\in\{1,2,3\}$ & \\
 $F\textnormal{\textuparrow}$ & upward irradiance  & \\
 $F\textnormal{\textdownarrow}$ & downward irradiance  & \\
 $i$, $j$, $k$  & integer indices & \\
 $K_{h}$ & turbulent diffusion coefficient for momentum & \\
 $K_{m}$ & turbulent diffusion coefficient for heat & \\
 $K_{1}$ & coefficient & \\
 $K_{2}$ & coefficient & \\
 $l$ & mixing length  & \\
 $L_{v}$ & heat of evaporation & $2.5\cdot 10^{6}\,\mathrm{J\,kg^{-1}}$ \\
 $LWP$ & liquid water path & \\
 $R$ & gas constant for dry air & $287\,\mathrm{J\,K^{-1}kg^{-1}}$\\
 $T$ & actual temperature & \\
 $T_{l}$ & actual liquid water temperature & \\
 $u$, $v$, $w$, $u_{i}$  & velocity components, $i\in\{1,2,3\}$ & \\
 $p_{0}$  & hydrostatic pressure & \\
 $q$ & total water content & \\
 $q_{l}$ & liquid water content & \\
 $q_{l_{\mathrm{crit}}}$ & threshold for the formation of precipitation & \\
 $q_{s}$ & specific humidity in case of saturation & \\
 $q_{v}$ & specific humidity & \\ 
 $x$, $y$, $z$, $x_{i}$  & Cartesian coordinates, $i\in\{1,2,3\}$ & \\
 $\Delta$ & characteristic grid length & \\
 $\epsilon$ & dissipation of sub-grid scale turbulent kinetic energy & \\
 $\varepsilon\textnormal{\textuparrow}$ & upward effective cloud emissivity  & \\
 $\varepsilon\textnormal{\textdownarrow}$ & downward effective cloud emissivity  & \\
 $\kappa$ &$R/c_{p}$  & 0.286 \\
 $\rho$ & air density  & \\
 $\tau$ & time scale for the Kessler scheme & \\
 $\theta$ & potential temperature & \\
 $\theta_{l}$ & liquid water potential temperature & \\
 $\theta_{v}$ & virtual potential temperature &  \\
 $\theta_{0}$ & reference value for the potential temperature & \\
 $\overline\psi$ & resolved scale variable & \\
 $\psi^{'}$ & sub-grid scale variable & \\
 $\psi^{\ast}$ & departure from the basic state (Boussinesq approximation) &  \\
 $\langle\psi\rangle$ & horizontal mean \\
\addlinespace
\bottomrule
\end{longtable}

\setlength\labelsep{0pt} 
\begin{bibliographie}{}
 \bibitem[Betts (1973)]{betts73}
   \textbf{Betts, A. K., 1973:} Non-precipitating cumulus convection and its parameterization.
   \textit{Quart. J. R. Meteorol. Soc.}, \textbf{99}, 178-196.
 \bibitem[Bougeault (1982)]{bougeault82}
   \textbf{Bougeault, P., 1982:} Modeling the trade-wind cumulus boundary layer. Part I: Testing the ensemble cloud relations 
   against numerical data. \textit{J. Atmos. Sci.}, \textbf{38}, 2414-2428.
 \bibitem[Cox (1976)]{cox76}
   \textbf{Cox, S. K., 1976:} Observations of cloud infrared effective emissivity.
   \textit{J. Atmos. Sci.}, \textbf{33}, 287-289.
 \bibitem[Cuijpers and Duynkerke (1993)]{cuijpers93}
   \textbf{Cuijpers, J. W. M. and P. G. Duynkerke, 1993:} Large eddy simulation of trade wind cumulus clouds.
   \textit{J. Atmos. Sci.}, \textbf{50}, 3894-3908.
 \bibitem[Deardorff (1976)]{deardorff76}
   \textbf{Deardorff, J. W., 1976:} Usefulness of liquid-water potential temperature in a shallow-cloud model. 
   \textit{J. Appl. Meteor.}, \textbf{15}, 98-102.
 \bibitem[Deardorff (1980)]{deardorff80}
   \textbf{Deardorff, J. W., 1980:} Stratocumulus-capped mixed layers derived from a three-dimensional model. 
   \textit{Boundary-Layer Meteorol.}, \textbf{18}, 495-527.
 \bibitem[Kessler (1969)]{kessler69}
   \textbf{Kessler, E., 1969:} On the distribution and continuity of water substance in atmospheric circulations. 
   \textit{Met. Monogr.}, \textbf{32}, 84 pp.
 \bibitem[M\"{u}ller and Chlond (1996)]{chlond96}
   \textbf{M\"{u}ller, G. A. and A. Chlond, 1996:} Three-dimensional numerical study of cell broadening during cold-air outbreaks. 
   \textit{Boundary-Layer Meteorol.}, \textbf{81}, 289-323.
 \bibitem[Ogura (1963)]{ogura63}
   \textbf{Ogura, Y., 1963:} The evolution of a moist convective element in a shallow, conditionally unstable atmosphere: 
   A numerical calculation. \textit{J. Atmos. Sci.}, \textbf{20}, 407-424.
 \bibitem[Orville (1965)]{orville65}
   \textbf{Orville, J. D., 1965:} A numerical study of the initiation of cumulus clouds over mountainous terrain. 
   \textit{J. Atmos. Sci.}, \textbf{22}, 684-699.
 \bibitem[Sommeria and Deardorff (1977)]{sommeria77}
   \textbf{Sommeria, G. and J. W. Deardorff, 1977:} Subgrid-scale condensation in models of nonprecipitating clouds.
   \textit{J. Atmos. Sci.}, \textbf{34}, 344-355.
 \bibitem[Stephans (1978)]{stephans78}
   \textbf{Stephans, G. L., 1978:} Radiation profiles in extended water clouds. II: Parameterization schemes. 
   \textit{J. Atmos. Sci.}, \textbf{35}, 2123-2132.
\end{bibliographie}

\end {document}