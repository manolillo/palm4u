\documentclass[11pt,a4paper,titlepage]{scrreprt}
\documentstyle[Flow]
\usepackage{graphics}
\usepackage{german}
\usepackage{chimuk}
\usepackage{bibgerm}
\usepackage{a4wide}
\usepackage{amsmath}
\usepackage{flafter}
\usepackage[dvips]{epsfig}
\usepackage{texdraw}
\usepackage{supertabular}
\usepackage{longtable}
\usepackage{scrpage}
\usepackage{fancyheadings}
\usepackage{flow}
\frenchspacing
\sloppy
\pagestyle{fancyplain}
\addtolength{\headheight}{\baselineskip}
\renewcommand{\chaptermark}[1]{\markboth{\thechapter~~#1}{}}
\renewcommand{\sectionmark}[1]{\markright{\thesection\ #1}}
\lhead{\fancyplain{}{\bfseries\leftmark}}
\rhead{}
%\begin{titlepage}
%\author{Gerald Steinfeld}
%\title{Prandtl layer parameterisation in PALM}
%\end{titlepage}
\begin{document}
%\maketitle
%\tableofcontents
\pagebreak

\chapter{Plant canopy model in PALM}
The basic equations used in the LES are the Navier-Stokes equations using the Boussinesq approximation (conservation of momentum; equation \ref{NSG}), 
the equation of continuity (conservation of mass; equation \ref{KON}), the first law of thermodynamics (conservation of energy; equation \ref{1HS}) and 
the conservation equation for a passive scalar (equation \ref{CPS}). All equations are filtered (in PALM by averaging over a discrete grid volume) in 
order to eliminate small-scale perturbations. Thus, the set of equations on that the LES model PALM is based is formed by the following equations:  

\begin{equation} \label{NSG}
\frac{\partial \overline{u}_i}{\partial t} = - \frac{\partial}{\partial x_k} \overline{u}_k \overline{u}_i - \frac{1}{\rho_0} \frac{\partial \overline{p^{\ast}}}{\partial x_i} - \left ( \epsilon_{ijk} f_j \overline{u}_k - \epsilon_{i3k} f_3 u_{g_k} \right ) + g \frac{\overline{\Theta^{\ast}}}{\Theta_0} \delta_{i3} - \frac{\partial}{\partial x_k} \overline{u_i'u_k'} - c_d a U \overline{u}_i
\end{equation}

\begin{equation} \label{KON} 
\frac{\partial \overline{u}}{\partial x_i} = 0
\end{equation}

\begin{equation} \label{1HS}
\frac{\partial \overline{\Theta}}{\partial t} = - \frac {\partial}{\partial x_k} \overline{u}_k \overline{\Theta} - \frac{\partial}{\partial x_k} \overline{u_k'\Theta'} + S
\end{equation}

\begin{equation} \label{CPS}
\frac{\partial \overline s}{\partial t} = - \frac{\partial}{\partial x_k} \overline{u}_k \overline{s} - \frac{\partial}{\partial x_k} \overline{u_k's'} - c_s a U \left ( \overline{s} - s_l \right )
\end{equation}

The overline denotes filtered variables, while the prime denotes sub-gridscale, local spatial fluctuations at time t that are not resolved by the model; 
the star denotes a departure of the basic state that is marked by a subscript $0$ and that varies only 
with height; $u_i$ is the velocity in the $x_i$-direction; $\rho$ is the air density; $p$ is the pressure; $f_i=(0,2\Omega cos(\phi),2\Omega sin(\phi))$ 
is the Coriolis parameter with the angular velocity of the earth $\Omega$; ${u_g}_i$ denotes the components of the geostrophic wind; $g$ is the 
acceleration due to gravity; $\Theta$ is the potential temperature and $s$ denotes the concentration of a passive scalar.
The sub-gridscale fluxes in equations \ref{NSG}, \ref{1HS} and \ref{CPS} are parameterised by a 1.5-order closure model using resolved-scale variables 
and the sub-gridscale turbulent kinetic energy $\overline{e}$. Thus, an additional prognostic equation has to be solved for the sub-gridscale turbulent 
kinetic energy $\overline{e}$:     

\begin{equation} \label{TKE}
\frac{\partial \overline e}{\partial t} = - \frac{\partial}{\partial x_k} \overline{u}_k \overline{e} - \tau_{ij} \frac{\partial \overline{u}_i}{\partial x_j} + \frac{g}{\Theta_0} \overline{u_3'\Theta_j'} - \frac{\partial}{\partial x_j} \left ( \overline{u_j' \left ( e + \frac{p'}{\rho_0} \right ) } \right ) - \epsilon - 2 c_d a U \overline{e}.
\end{equation}

Here, $\tau_{ij}$ denotes the sub-gridscale stress tensor. The final terms in equations \ref{NSG}, \ref{1HS}, \ref{CPS} and \ref{TKE} represent the sink 
or source of momentum, heat and the passive scalar due to canopy elements, respectively. Thus, these terms do only occur within the plant canopy, while 
they vanish outside the plant canopy (in PALM this distinction between canopy layer and the remaining area is realised by the specification of a height 
index that contains the information up to which vertical grid level the plant canopy extends; the additional terms for the plant canopy layer are 
only evaluated for grid levels smaller or equal that plant canopy height index). The additional, canopy related source and sink sources, are considered in 
PALM in the same way as in the models of WATANABE (2004, Boundary-Layer Meteorol., 112, 307-341) and SHAW AND SCHUMANN 
(1992, Boundary-Layer Meteorol., 61, 47-64).   
$c_d$ and $c_s$ are the leaf drag coefficient and the scalar exchange 
coefficient for a leaf, respectively; $a$ is the leaf area density; $U$ is the instantaneous local wind speed, defined as 
\begin{equation}
U = \left (  \overline{u}^2 + \overline{v}^2 + \overline{w}^2 \right )^{\frac{1}{2}},
\end{equation}
and $s_l$ is the scalar concentration at a leaf surface. 
The heat source distribution $S(z)$ in convective conditions has been given by BROWN AND COVEY (1966, Agric. Meteorol., 3, 73-96) and SHAW AND SCHUMANN 
(1992). The assumption that leads to the heat source distribution is that short-wave solar radiation penetrates the plant canopy and warms the foliage 
which, in turn, warms the air in contact with it. The heat source is distributed in such a way that the heat flux $Q$ at the top of the canopy, $z=h$, is 
prescribed and that the heat flux within the canopy follows a declining exponential function of the downward cumulative leaf area index. The decline of 
heat flux with decreasing height is similar to the decline of net radiation that has been described by BROWN AND COVEY (1966). The heat source $S(z)$ that 
is included in equation \ref{1HS} is the vertical derivative of the upward kinematic vertical heat flux $Q$   
\begin{equation}
S = \frac{\partial Q}{\partial z}.
\end{equation}
According to the explanations given above the upward kinematic vertical heat flux $Q$ is evaluated as
\begin{equation} \label{kvh}
Q(z) = Q(h) exp(- \alpha F(z)),
\end{equation}
where $F$ is the downward cumulative leaf area index (non-dimensional) determined by integration over the leaf area density $a$ 
\begin{equation}
F = \int_z^h a dz
\end{equation}
and $\alpha$ is an extinction coefficient that is set by default to 0.6 in PALM.

The final term in equation \ref{TKE} represents the sink for sub-gridscale turbulent kinetic energy 
due to the rapid dissipation of wake turbulence formed in the lee of canopy elements. This term is included 
on the assumption that individual wake motions behind canopy elements are of even smaller scale than those 
making up the bulk of sub-gridscale kinetic energy. For simulations in that the grid length is of the 
order of the scale of wake turbulence modifications within equation \ref{TKE} might be necessary. KANDA 
AND HINO (1994, Boundary-Layer Meteorol., 68, 237-257) made a suggestion how equation \ref{TKE} has to be 
modified for such cases. 

\end{document}
